\documentclass[12pt]{amsart}
\usepackage{hyperref}
\title{Problem 4.5.6 p96} %put the reference to the CSP problem
\author{Ngoc Tran, princengoc@github} %put your name and your github id
\date{October 15th, 2015} %use US date format

\theoremstyle{plain}
\newtheorem{theorem}{Theorem}[section]
\newtheorem{lemma}[theorem]{Lemma}
\newtheorem{corollary}[theorem]{Corollary}

\begin{document}
\maketitle

\section{The problem} 
%recall the problem, so that we know which problem in the text you are referring to.
%please try to be as self-contained as possible, in the context of CSP
Problem 4.5.6 in \cite[page 96]{MR2245368}, is on the existence of $\alpha$-fragmentation. 
It is shown in \cite[Chapter 9]{MR2245368} that for $\alpha = 1/2$, there is a natural construction of $\Pi_\infty(\lambda), \lambda \geq 0)$ as a partition valued fragmentation process, meaning that $\Pi_\infty(\lambda)$ is constructed for each $\lambda$ on the same probability space, in such a way that $\Pi_\infty(\lambda)$ is a coarser partition than $\Pi_\infty(\mu)$ whenever $\lambda < \mu$. Does there exist a similar construction for index $\alpha \neq 1/2$? 

\section{Literature review} %put a mini review of the problem: for example, motivations, current status, thoughts, ...  
A natural guess is that
such a construction might be made with one of the self-similar fragmentation
processes of Bertoin \cite{MR1899456}, but Miermont and Schweinsberg \cite{MR2053052} have shown that
a construction of this form is possible only for $\alpha = 1/2$. 

\section{Status} %choose one of the three: open, solved, partially solved.
Open.

\bibliographystyle{plain}
\bibliography{problem456p96,csp} %bib file name equals the document title, without special characters. 
%always include csp.bib, which contains the standard bibtex format of the book Combinatorial Stochastic Processes.

\end{document}
